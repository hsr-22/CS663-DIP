\documentclass{article}
\usepackage{amsmath}
\usepackage{graphicx}
\usepackage{float}
\usepackage{hyperref}
\usepackage{fancyvrb}
\usepackage{enumitem}
\usepackage{matlab-prettifier}
\setlength{\parindent}{0pt}
\graphicspath{{../images/}}

\title{CS663: Digital Image Processing - Homework 5}
\author{Harsh $\vert$ Pranav $\vert$ Swayam} 
\date{November 6, 2024}

\begin{document}

\maketitle
\flushleft
\section*{Homework 5 - Question 4}

\subsection*{Part a - Reconstruction}

Since we know the locations of the non-zero elements, we can formulate this reconstruction problem as follows:

   
   The DFT of an \( n \times n \) image \( f(x, y) \) at a frequency \( (u, v) \) is given by:
   \[
   F(u, v) = \sum_{x=0}^{n-1} \sum_{y=0}^{n-1} f(x, y) e^{-j 2 \pi \left( \frac{ux}{n} + \frac{vy}{n} \right)}.
   \]
   Since \( f(x, y) \) is sparse, we can simplify this sum to only include the non-zero locations:
   \[
   F(u, v) = \sum_{(x_i, y_i) \in \Omega} f(x_i, y_i) e^{-j 2 \pi \left( \frac{ux_i}{n} + \frac{vy_i}{n} \right)},
   \]
   where \( \Omega \) is the set of known locations with non-zero values, and \( |\Omega| = k \).

   
   The non-zero values in \( f(x, y) \) as a vector 
   \[ f_\Omega = [f(x_1, y_1), f(x_2, y_2), \ldots, f(x_k, y_k)]^T \]
   
   Let \( F_m \) represent the vector of \( m \) observed DFT coefficients at frequencies \( (u_i, v_i) \) for \( i = 1, 2, \dots, m \).
   We can set up the following linear system:
   \[
   F_m = A f_\Omega,
   \]
   where \( A \) is an \( m \times k \) matrix with entries \( A_{ij} = e^{-j 2 \pi \left( \frac{u_i x_j}{n} + \frac{v_i y_j}{n} \right)} \).


   If \( m \geq k \), we have enough equations to uniquely determine the \( k \) unknown values in \( f_\Omega \).
   If \( m = k \) and \( A \) has full column rank (i.e., the rows of \( A \) are linearly independent), then \( A \) is invertible, and we can solve for \( f_\Omega \) directly:
   \[
   f_\Omega = A^{-1} F_m.
   \]
   Alternatively, if \( m > k \), we can use a least-squares approach to solve for \( f_\Omega \), which provides robustness against noise and dependencies between Fourier coefficients.

\subsection*{Part b - Minimum Value of \( m \) Needed for Reconstruction}

% \subsubsection*{Argument for \( m = k \): Theoretical Minimum}
\begin{itemize}
    \item Uniqueness Condition:
    
    Since we have \( k \) unknowns in \( f_\Omega \), the minimum number of independent equations needed to uniquely determine these unknowns is \( k \).
    
    Thus, \textbf{in an ideal, noise-free scenario}, if we collect exactly \( m = k \) DFT coefficients and ensure they are independent, we can reconstruct \( f_\Omega \) by solving a system of \( k \) equations in \( k \) unknowns.

    \item Mathematical Justification:

    With \( m = k \), the system \( F_m = A f_\Omega \) becomes exactly determined. If \( A \) has full rank, this system has a unique solution for \( f_\Omega \).
    
    Therefore, \textbf{\( m = k \) is the minimum number of DFT coefficients required} for unique reconstruction of \( f(x, y) \) in this idealized setting.

\end{itemize}

\subsection*{Part c - Unknown Locations of Non-Zero Elements}

No, as multiple configurations of non-zero elements can yield identical DFT coefficients. For instance, if two different sets of pixels have the same Fourier representation due to their positions and values being interchanged.

\end{document}