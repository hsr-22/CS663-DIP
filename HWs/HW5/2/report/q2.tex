\documentclass{article}
\usepackage{amsmath}
\usepackage{graphicx}
\usepackage{float}
\usepackage{hyperref}
\usepackage{fancyvrb}
\usepackage{enumitem}
\usepackage{matlab-prettifier}
\setlength{\parindent}{0pt}
\graphicspath{{../images/}}

\title{CS663: Digital Image Processing - Homework 5}
\author{Harsh $\vert$ Pranav $\vert$ Swayam} 
\date{November 6, 2024}

\begin{document}

\maketitle
\flushleft
\section*{Homework 5 - Question 2}

Given the images \( g_1 \) and \( g_2 \) taken under different focus settings, we are expressing them mathematically as:

\begin{itemize}
    \item \( g_1 = f_1 + h_2 * f_2 \), where:
        \begin{itemize}
            \item \( f_1 \) is the scene outside (in focus for \( g_1 \)),
            \item \( h_2 \) is the blur kernel applied to the reflection \( f_2 \),
            % \item \( * \) denotes convolution.
        \end{itemize}
    \item \( g_2 = h_1 * f_1 + f_2 \), where:
        \begin{itemize}
            \item \( h_1 \) is the blur kernel applied to the outside scene \( f_1 \) when the reflection \( f_2 \) is in focus.
        \end{itemize}
\end{itemize}

\subsection*{Fourier Transform Approach}

To solve for \( f_1 \) and \( f_2 \), let's take the Fourier Transform of both equations:

\[
G_1(\mu) = F_1(\mu) + H_2(\mu) F_2(\mu)
\]
\[
G_2(\mu) = H_1(\mu) F_1(\mu) + F_2(\mu)
\]

Now we can rearrange these equations to isolate \( F_1(\mu) \) and \( F_2(\mu) \):

\begin{enumerate}
    \item Solving for \( F_2(\mu) \):
    \[
    F_2(\mu) = \frac{G_2(\mu) - H_1(\mu) G_1(\mu)}{1 - H_1(\mu) H_2(\mu)}
    \]
    \item Solving for \( F_1(\mu) \):
    \[
    F_1(\mu) = \frac{G_1(\mu) - H_2(\mu) G_2(\mu)}{1 - H_1(\mu) H_2(\mu)}
    \]
\end{enumerate}

\subsection*{Observations on the Formula}

This solution is defined as long as \( 1 - H_1(\mu) H_2(\mu) \neq 0 \). However, due to the nature of \( h_1 \) and \( h_2 \) as low-pass blur kernels, each integrates to 1 over the entire domain (i.e., \( H_1(0) = H_2(0) = 1 \)). This implies that at the frequency \( \mu = 0 \), we encounter:

\[
1 - H_1(0) H_2(0) = 0
\]

As a result, the formula becomes undefined at low frequencies, particularly at the DC component (zero frequency). This introduces an issue where we can recover high-frequency details in the images, but we cannot reliably reconstruct low-frequency components.

Hence, to correct this we can add a small \( \epsilon \) to the denominator to stabilize the solution:

\[
F_2(\mu) = \frac{G_2(\mu) - H_1(\mu) G_1(\mu)}{1 - H_1(\mu) H_2(\mu) + \epsilon}
\]
\[
F_1(\mu) = \frac{G_1(\mu) - H_2(\mu) G_2(\mu)}{1 - H_1(\mu) H_2(\mu) + \epsilon}
\]

\end{document}