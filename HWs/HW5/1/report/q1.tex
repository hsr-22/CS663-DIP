\documentclass{article}
\usepackage{amsmath}
\usepackage{graphicx}
\usepackage{float}
\usepackage{hyperref}
\usepackage{fancyvrb}
\usepackage{enumitem}
\usepackage{matlab-prettifier}
\setlength{\parindent}{0pt}
\graphicspath{{../images/}}

\title{CS663: Digital Image Processing - Homework 5}
\author{Harsh $\vert$ Pranav $\vert$ Swayam} 
\date{November 6, 2024}

\begin{document}

\maketitle
\flushleft
\section*{Homework 5 - Question 1}

% The problem involves maximizing a quadratic form subject to constraints, using the method of Lagrange multipliers. The matrix $C$ is symmetric, and we aim to find the eigenvectors and eigenvalues of $C$. The solution proceeds in two parts: maximizing $J_1(f)$ and $J_2(g)$.

\subsection*{Translation}
Let us assume an initial image $f_1(x, y)$ of size $N \times N$. Then, by translating the image by $(x_0, y_0)$, we obtain a new image $f_2(x, y) = f_1(x - x_0, y - y_0)$. 

The Fourier transforms of these images would then be related as
\[
    F_2(\mu, \nu) = e^{-j 2 \pi (\mu x_0 + \nu y_0)} \times F_1(\mu, \nu)
\]
The cross-power spectrum $C(\mu, \nu)$ of both images is then given by:
\[
    C(\mu, \nu) = \frac{F_2^*(\mu, \nu) F_1(\mu, \nu)}{|F_2(\mu, \nu)||F_1(\mu, \nu)|} = e^{j 2 \pi (\mu x_0 + \nu y_0)}
\]
Taking the inverse Fourier transform of $C(\mu, \nu)$, we obtain a delta function that is centered at $(-x_0, -y_0)$:
\[
    F^{-1} \left( e^{j 2 \pi (\mu x_0 + \nu y_0)} \right) = \delta(x + x_0, y + y_0)
\]

Through this, we are able to obtain the shift between the two images (\underbar{i.e.} $(x_0, y_0)$).

\subsubsection*{Time Compexity}
The above method involves computing the Fourier transform of $N \times N$ images and the inverse Fourier transform of the cross-power spectrum,
 which have time complexity of $O(N^2 \log N)$ via FFT.
 The computation of the cross-power spectrum has a time complexity of $O(N^2)$.
\newline
\newline
The overall time complexity is thus \(\mathbf{O(N^2 \log N)}\).
\newline
\newline
A pixel-wise approach would have a time complexity of \(O(N^2W^2)\) (\(W\) being the window size) with the worst case being \(O(N^4)\).
This is significantly higher than the FFT-based approach.
\pagebreak
\subsection*{Rotation}
Let us again assume an initial image $f_1(x, y)$ of size $N \times N$. Then, by translating the image by $(x_0, y_0)$ and rotated by an angle $\theta_0$
, we obtain a new image $ f_2(x, y) = f_1(x \cos \theta_0 + y \sin \theta_0 - x_0, -x \sin \theta_0 + y \cos \theta_0 - y_0)$. 

The Fourier transforms of these images would then be related as
\[
    F_2(\mu, \nu) = e^{-j 2 \pi (\mu x_0 + \nu y_0)} \times F_1(\mu \cos \theta_0 + \nu \sin \theta_0, -\mu \sin \theta_0 + \nu \cos \theta_0)
\]

With the same logic as before, the cross-power spectrum $C(\mu, \nu)$ of both images can be used to find the shift $(x_0, y_0)$.
\newline
\newline
To find the rotation angle $\theta_0$, we first take the magnitude of the Fourier transforms of both images, thereby obtaining:
\[
    M_2(\mu, \nu) = M_1(\mu \cos \theta_0 + \nu \sin \theta_0, -\mu \sin \theta_0 + \nu \cos \theta_0)
\]

A rotation in the rectangular coordinates corresponds to a transation in the polar coordinates. 
\[
\rho = \sqrt{\mu^2 + \nu^2}
\]
\[
\theta_1 = \arctan\left(\frac{\nu}{\mu}\right)
\]
\[
\theta_2 = \arctan\left(\frac{\nu \cos \theta_0 - \mu \sin \theta_0}{\mu \cos \theta_0 + \nu \sin \theta_0}\right) 
= \arctan\left(\frac{\nu}{\mu}\right) - \theta_0
\]

Thus, the rotation angle can be found by finding the shift in the polar coordinates, in the same way as the translation.
\[
M_2(\rho, \theta) = M_1(\rho, \theta +  \theta_0)
\]
\end{document}