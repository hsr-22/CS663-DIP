\documentclass{article}
\usepackage{amsmath}
\usepackage{amssymb}
\usepackage{graphicx}
\usepackage{float}
\usepackage{hyperref}
\usepackage{fancyvrb}
\usepackage{matlab-prettifier}
\setlength{\parindent}{0pt}
\graphicspath{{../images/}}

\title{CS663: Digital Image Processing - Homework 3}
\author{Harsh $\vert$ Pranav $\vert$ Swayam} 
\date{October 1, 2024}

\begin{document}

\maketitle
\section{Homework 3 - Question 6}

First we prove \( F(F(f(t))) = f(-t) \):

\begin{align*}
    G(f) = F(f(t)) &= \int_{-\infty}^{\infty} f(t) e^{-j 2 \pi f t} dt \\
    H(\tau) = F(G(f)) = F(F(f(t))) &= \int_{-\infty}^{\infty} G(f) e^{-j 2 \pi f \tau} df \\
    &= \int_{-\infty}^{\infty} \left( \int_{-\infty}^{\infty} f(t) e^{-j 2 \pi f t} dt \right) e^{-j 2 \pi f \tau} df \\
    &= \int_{-\infty}^{\infty} f(t) \left( \int_{-\infty}^{\infty} e^{-j 2 \pi f (t + \tau)} df \right) dt \\ &(\because\text{ f and t are independent}) \\
    &= \int_{-\infty}^{\infty} f(t) \delta(t + \tau) dt \quad \text{(Shifting)} \\
    &= f(-\tau) \quad \text{(Sifting)}
\end{align*}

$\therefore H(\tau) = f(-\tau) \implies H(t) = f(-t) \implies F(F(f(t))) = f(-t)$

Now, if \( F(F(f(t))) = f(-t) \), then 

$F(F(F(F(f(t))))) = F(F(f(-t))) = f(-(-t)) = f(t)$

% A practical use of this property is in the context of image processing. For example, if we have an image and we apply a filter to it, we can apply the filter again to the output of the first filter to get the original image back. This can be useful in cases where we want to apply a filter to an image and then apply the inverse of the filter to the output to get the original image back.

One very important application of this is that it offers a reasoning for the construction of the inverse fourier transform.
Operations in the spatial domain affect the frequency domain and vice versa. For instance, applying a low-pass or high-pass filter in the frequency domain affects the smoothness or edge characteristics of an image in the spatial domain. The reverse holds due to duality.

\end{document}