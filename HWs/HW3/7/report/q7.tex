\documentclass{article}
\usepackage{amsmath}
\usepackage{graphicx}
\usepackage{float}
\usepackage{hyperref}
\usepackage{fancyvrb}
\usepackage{matlab-prettifier}
\setlength{\parindent}{0pt}
\graphicspath{{../images/}}

\title{CS663: Digital Image Processing - Homework 3}
\author{Harsh $\vert$ Pranav $\vert$ Swayam} 
\date{October 1, 2024}

\begin{document}

\maketitle
\section{Homework 3 - Question 7}

% \subsection*{The Isotropic Heat Equation and Its Fourier Transform}

The given partial differential equation (PDE) is:

\[
\frac{\partial I}{\partial t} = c \left( \frac{\partial^2 I}{\partial x^2} + \frac{\partial^2 I}{\partial y^2} \right)
\]

This is the isotropic heat equation, where \( I(x, y, t) \) represents an image (or a function) that evolves over time, and \( c \) is a non-negative constant.

\subsection*{Fourier Transform of the Heat Equation}

To solve this problem, we apply the Fourier transform to both sides of the PDE. Using the Fourier differentiation theorem, which states that the Fourier transform of a derivative of a function becomes a multiplication by a polynomial in frequency, we transform the equation in the spatial domain into the frequency domain.

Let the Fourier transform of \( I(x, y, t) \) be denoted as \( \hat{I}(u, v, t) \), where \( u \) and \( v \) are the Fourier domain variables corresponding to the spatial variables \( x \) and \( y \), respectively.

\subsubsection*{Applying Fourier Transform to Both Sides of the PDE}

The heat equation in the spatial domain is:

\[
\frac{\partial I(x, y, t)}{\partial t} = c \left( \frac{\partial^2 I(x, y, t)}{\partial x^2} + \frac{\partial^2 I(x, y, t)}{\partial y^2} \right)
\]

Now, we take the 2D Fourier transform of both sides.
The Fourier transform of the time derivative \( \frac{\partial I(x, y, t)}{\partial t} \) is:
  \[
  \mathcal{F}\left\{\frac{\partial I(x, y, t)}{\partial t}\right\} = \frac{\partial \hat{I}(u, v, t)}{\partial t}
  \]
Using the differentiation property of Fourier transforms, the Fourier transform of the second-order derivative \( \frac{\partial^2 I(x, y, t)}{\partial x^2} \) is:
  \[
  \mathcal{F}\left\{ \frac{\partial^2 I(x, y, t)}{\partial x^2} \right\} = (j 2 \pi u)^2 \hat{I}(u, v, t)
  \]
Similarly, the Fourier transform of \( \frac{\partial^2 I(x, y, t)}{\partial y^2} \) is:
  \[
  \mathcal{F}\left\{ \frac{\partial^2 I(x, y, t)}{\partial y^2} \right\} = (j 2 \pi v)^2 \hat{I}(u, v, t)
  \]
Substituting these into the Fourier-transformed PDE, we get:

\[
\frac{\partial \hat{I}(u, v, t)}{\partial t} = -c \left( (2 \pi u)^2 + (2 \pi v)^2 \right) \hat{I}(u, v, t)
\]

This is a first-order differential equation in time for \( \hat{I}(u, v, t) \). We can solve it by separating variables:
\[
\frac{1}{\hat{I}(u, v, t)} \frac{\partial \hat{I}(u, v, t)}{\partial t} = -c \left( (2 \pi u)^2 + (2 \pi v)^2 \right)
\]
Integrating both sides with respect to \( t \):
\[
\ln(\hat{I}(u, v, t)) = -c \left( (2 \pi u)^2 + (2 \pi v)^2 \right) t + \ln(\hat{I}(u, v, 0))
\]
Exponentiation on both sides, we get:
\[
\hat{I}(u, v, t) = \hat{I}(u, v, 0) \cdot e^{-c \left( (2 \pi u)^2 + (2 \pi v)^2 \right) t}
\]
Here, \( \hat{I}(u, v, 0) \) is the initial Fourier transform of the image at \( t = 0 \).

% \subsubsection*{Identify the Gaussian in the Fourier Domain}

We recognize that the solution in the frequency domain has the form of the Fourier transform of a Gaussian. The exponential term \( e^{-c \left( (2 \pi u)^2 + (2 \pi v)^2 \right) t} \) is the Fourier transform of a Gaussian in the spatial domain.
\vspace{5pt}

We know that the fourier transform of a Gaussian is also a Gaussian, and the inverse fourier transform of the Gaussian in frequency domain will give a Gaussian in the spatial domain.
\vspace{5pt}

Thus, the solution in the spatial domain \( I(x, y, t) \) is obtained by convolving the initial image \( I(x, y, 0) \) with a Gaussian. The standard deviation of this Gaussian depends on the parameters of the exponential term.

The Fourier transform of the Gaussian in the frequency domain is:

\[
e^{-\frac{(2\pi\sigma)^2}{2} \left( u^2 + v^2 \right)}
\]

Comparing this with our result \( e^{-c \left( (2 \pi u)^2 + (2 \pi v)^2 \right) t} \), we can identify the standard deviation \( \sigma \) by matching terms. Specifically:

\[
\frac{\sigma^2}{2} = c t
\]

Solving for \( \sigma \), we get:

\[
\sigma = \sqrt{2 c t}
\]

\subsubsection*{Conclusion}

Thus, running the isotropic heat equation on an image is equivalent to convolving the image with a Gaussian of zero mean and standard deviation:

\[
\sigma = \sqrt{2 c t}
\]

This shows that the heat equation diffuses the image over time, and the amount of diffusion (i.e., the standard deviation of the Gaussian) increases as \( t \) increases.

\end{document}