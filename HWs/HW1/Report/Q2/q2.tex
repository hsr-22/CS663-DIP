\documentclass{article}
\usepackage{amsmath}
\usepackage{graphicx}
\usepackage{float}
\usepackage{hyperref}

\title{CS663: Digital Image Processing - Homework 1}
\author{Harsh $\vert$ Pranav $\vert$ Swayam} 

\begin{document}

\maketitle
\section{Homework 1 - Question 2}


\subsection*{Given:}

$u_{12}$ represents the motion from $I_1$ to $I_2$ and a similar relation for $u_{23}$ and $u_{13}$.

\subsection*{Relation:}
$$u_{13}=u_{12}+u_{23}$$


\subsubsection*{Explanation:} 

Since the motion is solely transnational therefore the total displacement from $I_1$ to $I_3$ should be the sum of the displacement between $I_1$-$I_2$ and $I_2$-$I_3$.


\subsection*{Practicality:}


Some troubles may arise when this relations is used in practical. The reasons may be:


\subsubsection*{Numerical Approximations:} 

Numerical methods used in computation introduce rounding errors, which can accumulate and cause the computed motion vectors to deviate slightly from the theoretical relationship.


\subsubsection*{Noise:} 

The camera and other similar devices may be cursed with random noise, which can distort and hamper some features leading to small errors.

\end{document}