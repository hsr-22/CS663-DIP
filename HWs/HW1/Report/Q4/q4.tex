\documentclass{article}
\usepackage{amsmath}
\usepackage{graphicx}
\usepackage{float}
\usepackage{hyperref}

\title{CS663: Digital Image Processing - Homework 1}
\author{Harsh $\vert$ Pranav $\vert$ Swayam} 

\begin{document}

\maketitle
\section{Homework 1 - Question 4}

To perform motion estimation using control points for the given motion model, we need to estimate the unknown constants \(a, b, c, d, e, f, A, B, C, D, E, F\).

\subsection*{Given}
The motion models:
\[
x_2 = ax_1^2 + by_1^2 + cx_1y_1 + dx_1 + ey_1 + f
\]
\[
y_2 = Ax_1^2 + By_1^2 + Cx_1y_1 + Dx_1 + Ey_1 + F
\]
Here, \( (x_1, y_1) \) are the coordinates of a point in the first image, and \( (x_2, y_2) \) are the coordinates of the corresponding point in the second image. The goal is to estimate the coefficients \( a, b, c, d, e, f, A, B, C, D, E, F \) using a set of known control points.

\subsection*{System of Equations}
Suppose we have \(N\) control points, where each control point in Image 1 is denoted as \( (x_{1i}, y_{1i}) \) and the corresponding point in Image 2 as \( (x_{2i}, y_{2i}) \) for \( i = 1, 2, \ldots, N \).

For each control point, we have two equations:

\[
x_{2i} = a x_{1i}^2 + b y_{1i}^2 + c x_{1i}y_{1i} + d x_{1i} + e y_{1i} + f
\]
\[
y_{2i} = A x_{1i}^2 + B y_{1i}^2 + C x_{1i}y_{1i} + D x_{1i} + E y_{1i} + F
\]

\subsection*{Matrix Form}
For the \(x_2\) coordinate:
\[
\begin{bmatrix}
x_{21} \\
x_{22} \\
\vdots \\
x_{2N}
\end{bmatrix}
=
\begin{bmatrix}
x_{11}^2 & y_{11}^2 & x_{11}y_{11} & x_{11} & y_{11} & 1 \\
x_{12}^2 & y_{12}^2 & x_{12}y_{12} & x_{12} & y_{12} & 1 \\
\vdots & \vdots & \vdots & \vdots & \vdots & \vdots \\
x_{1N}^2 & y_{1N}^2 & x_{1N}y_{1N} & x_{1N} & y_{1N} & 1
\end{bmatrix}
\begin{bmatrix}
a \\
b \\
c \\
d \\
e \\
f
\end{bmatrix}
\]
This can be simplified as:
\[
\mathbf{x}_2 = \mathbf{X}_1 \cdot \mathbf{p}
\]
where
\[
\mathbf{x}2 = \begin{bmatrix} x{21} \\ x_{22} \\ \vdots \\ x_{2N} \end{bmatrix}, \quad
\mathbf{X}_1 = \begin{bmatrix}
x_{11}^2 & y_{11}^2 & x_{11}y_{11} & x_{11} & y_{11} & 1 \\
x_{12}^2 & y_{12}^2 & x_{12}y_{12} & x_{12} & y_{12} & 1 \\
\vdots & \vdots & \vdots & \vdots & \vdots & \vdots \\
x_{1N}^2 & y_{1N}^2 & x_{1N}y_{1N} & x_{1N} & y_{1N} & 1
\end{bmatrix}, \quad
\mathbf{p} = \begin{bmatrix} a \\ b \\ c \\ d \\ e \\ f \end{bmatrix}
\]

Similarly, for the \(y_2\) coordinate:
\[
\mathbf{y}_2 = \mathbf{X}_1 \cdot \mathbf{P}
\]
where
\[
\mathbf{y}2 = \begin{bmatrix} y{21} \\ y_{22} \\ \vdots \\ y_{2N} \end{bmatrix}, \quad
\mathbf{P} = \begin{bmatrix} A \\ B \\ C \\ D \\ E \\ F \end{bmatrix}
\]

\subsection*{Solving the System of Equations}
To estimate the parameters \( \mathbf{p} = [a, b, c, d, e, f]^T \) and \( \mathbf{P} = [A, B, C, D, E, F]^T \), we solve the linear systems for \( \mathbf{p} \) and \( \mathbf{P} \):

\[
\mathbf{p} = \mathbf{X}_1^\dagger \mathbf{x}_2
\]
\[
\mathbf{P} = \mathbf{X}_1^\dagger \mathbf{y}_2
\]
where \( \mathbf{X}^\dagger = (\mathbf{X}^T \mathbf{X})^{-1} \mathbf{X}^T
\) of \( \mathbf{X} \). This is used here because \( \mathbf{X}_1 \) might not be a square matrix.

\subsection*{Finally}
After solving these systems, the vectors \( \mathbf{p} \) and \( \mathbf{P} \) contain the estimated coefficients that describe the motion between the two images. These coefficients can then be used to map any point \( (x_1, y_1) \) in Image 1 to its corresponding location \( (x_2, y_2) \) in Image 2 according to the given motion model.


\end{document}