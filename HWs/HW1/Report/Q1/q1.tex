\documentclass{article}
\usepackage{amsmath}
\usepackage{graphicx}
\usepackage{float}
\usepackage{hyperref}
\setlength{\parindent}{0pt}

\title{CS663: Digital Image Processing - Homework 1}
\author{Harsh $\vert$ Pranav $\vert$ Swayam} 

\begin{document}

\maketitle
\section{Homework 1 - Question 1}

In this question, we are given two images of the same scene, but with different pixel sizes. But, for image alignment purposee, the size of the pixels should not be of importance, since transformation would depend on the pixel coordinates. Thus, image size should not affect the alignment process.

The original pixel size of the first image is $0.5 \times 0.5$, both sizes in mm.

\vspace{10pt}
For the first case, when the second image has a pixel size of $0.25 \times 0.25$, the new coordinates, which we represent as $(x', y')$, can be obtained from the original coordinates $(x, y)$ as follows:

\begin{equation}
\begin{bmatrix}
x' \\
y' 
\end{bmatrix}
=
\begin{bmatrix}
0.5 & 0 \\
0 & 0.5 \\
\end{bmatrix}
\begin{bmatrix}
x \\
y 
\end{bmatrix}
\end{equation}

Thus, the motion model that we can adopt to attain the desired alignment is the rigid (rotation + translation) model which will be followed by a scaling transformation in both X and Y directions based on the matrix given above.

\hrulefill
\vspace{5pt}

For the second case, when the second image has a pixel size of $0.25 \times 0.5$, the new coordinates, which we represent as $(x_2', y_2')$, can be obtained from the original coordinates $(x, y)$ as follows:

\begin{equation}
\begin{bmatrix}
x_2' \\
y_2'
\end{bmatrix}
=
\begin{bmatrix}
0.5 & 0 \\
0 & 1 \\
\end{bmatrix}
\begin{bmatrix}
x \\
y
\end{bmatrix}
\end{equation}

Thus, the motion model that we can adopt to attain the desired alignment is the rigid (rotation + translation) model which will be followed by a scaling transformation in {\bf only the X} direction based on the matrix given above.

% Answer : 

% There are two possibilties - one is that the size of the image scales down with the pixel size (simple compression), while the other is that the size of the image itself does not change implying a difference in resolution.

% In the first case, the motion model that we would adopt is a simple scaling model. This is because the image is simply scaled down and the only transformation that is required is a scaling transformation. The scaling transformation is given by the following equation:

% \begin{equation}
% \begin{bmatrix}
% x' \\
% y' \\
% 1
% \end{bmatrix}
% =
% \begin{bmatrix}
% s_x & 0 & 0 \\
% 0 & s_y & 0 \\
% 0 & 0 & 1
% \end{bmatrix}
% \begin{bmatrix}
% x \\
% y \\
% 1
% \end{bmatrix}
% \end{equation}

% where $s_x$ and $s_y$ are the scaling factors in the x and y directions respectively. In this case, $s_x = 0.5/0.25 = 2$ and $s_y = 0.5/0.25 = 2$.

% In the second case, 


\end{document}